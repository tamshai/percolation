%%% Edit ONLY the abstract text block that corresponds to the language!

{\sffamily

\noindent
\if\mylang e
\includegraphics[height=30mm]{Aalto_SCI_EN}
\fi
\if\mylang f
\includegraphics[height=30mm]{Aalto_SCI_FI}
\fi
\if\mylang s
\includegraphics[height=30mm]{Aalto_SCI_SE}
\fi

\vspace*{-21mm}

{\small
{\bfseries
\if\mylang e
\begin{tabular*}{105mm}{lr}
\multicolumn{1}{l}{\mbox{}\hspace{65mm}\mbox{}}
& \textcolor{gray}{Aalto University, P.O. BOX 11000, 00076 AALTO} \\
& \textcolor{gray}{www.aalto.fi} \\
& Abstract of Physics Special Assignment \\
\end{tabular*}
\fi
\if\mylang f
\begin{tabular*}{105mm}{lr}
\multicolumn{1}{l}{\mbox{}\hspace{80mm}\mbox{}}
& \textcolor{gray}{Aalto-yliopisto, PL 11000, 00076 AALTO} \\
& \textcolor{gray}{www.aalto.fi} \\
& Fysiikan erikoistyön tiivistelmä \\
\end{tabular*}
\fi
\if\mylang s
\begin{tabular*}{105mm}{lr}
\multicolumn{1}{l}{\mbox{}\hspace{70mm}\mbox{}}
& \textcolor{gray}{Aalto-universitetet, PB 11000, 00076 AALTO} \\
& \textcolor{gray}{www.aalto.fi} \\
& Sammandrag av specialarbete i fysik \\
\end{tabular*}
\fi
}
}

\vspace*{10mm}

\if\mylang e
\begin{tabular*}{160mm}{lll}
\hline
\multicolumn{3}{l}{\textbf{Author} \myauthor} \\[1mm] \hline
\multicolumn{3}{l}{\textbf{Title of the work} \mytitle}  \\[1mm] \hline
\multicolumn{3}{l}{\textbf{Degree programme} \mydegreeprog} \\[1mm] \hline
\multicolumn{2}{l}{\textbf{Major} \mymajor} &
\textbf{Code of major} \mymajorcode \\[1mm] \hline
\multicolumn{3}{l}{\textbf{Supervisor} \mysupervisor} \\[1mm] \hline
\multicolumn{3}{l}{\textbf{Instructor(s)} \myinstructor} \\[1mm] \hline
\textbf{Date} \mypubdate &
\textbf{Number of pages} \mypagenum &
\textbf{Language} \mylanguage \\[1mm] \hline \\[1mm]
\textbf{Abstract} \\[1mm]
\multicolumn{3}{p{155mm}}{%
%%%
%%% START ABSTRACT TEXT (ENGLISH REPORT)
%%%
We derive a formalism to order nuclides by their importance to a given quantity of 
interest. A quantity can be any result of any linear effect of nuclides. Linear effects 
yield quantities that depend linearly on the amounts of nuclides, such as decay heat 
emission rate, spontaneous fission rate and neutron balance rate. Non-linear effects yield 
quantities such as effective multiplication factor and detectability of nuclides, and 
depend non-linearly on the amounts of nuclides.

The formalism can be used in identification of important minority and majority nuclides 
for any application. Application-specific criteria must be expressed as quantities of 
interest for which effects must be derived. This part remains to be an art. The formalism 
can be used to order nuclides by their importance to selected quantities of interest. 
Another art is to judge which nuclide is the last important one for a quantity: any 
nuclide with less importance is not important.

Suitable criteria, expressed as quantities of interest, are presented for three 
applications: reactor performance, radioactive waste storage and proliferation resistance. 
For illustration, the formalism is applied to a case of radioactive waste storage of a 
high burnup BWR assembly.
%%%
%%% END ABSTRACT TEXT
%%%
}
\\[1mm] \hline
\multicolumn{3}{l}{\textbf{Keywords} \mykeywords} \\[1mm] \hline
\end{tabular*}
\fi

\if\mylang f
\begin{tabular*}{160mm}{lll}
\hline
\multicolumn{3}{l}{\textbf{Tekijä} \myauthor} \\[1mm] \hline
\multicolumn{3}{l}{\textbf{Työn nimi} \mytitle}  \\[1mm] \hline
\multicolumn{3}{l}{\textbf{Koulutusohjelma} \mydegreeprog} \\[1mm] \hline
\multicolumn{2}{l}{\textbf{Pääaine} \mymajor} &
\textbf{Pääaineen koodi} \mymajorcode \\[1mm] \hline
\multicolumn{3}{l}{\textbf{Työn valvoja} \mysupervisor} \\[1mm] \hline
\multicolumn{3}{l}{\textbf{Työn ohjaaja(t)} \myinstructor} \\[1mm] \hline
\textbf{Päivämäärä} \mypubdate &
\textbf{Sivumäärä} \mypagenum &
\textbf{Kieli} \mylanguage \\[1mm] \hline
\textbf{Tiivistelmä} \\[1mm] \\[1mm]
\multicolumn{3}{p{155mm}}{%
%%%
%%% START ABSTRACT TEXT (FINNISH REPORT)
%%%
Tähän kirjoitetaan suomenkielisen erikoistyön tiivistelmä.
%%%
%%% END ABSTRACT TEXT
%%%
}
\\[1mm] \hline
\multicolumn{3}{l}{\textbf{Avainsanat} \mykeywords} \\[1mm] \hline
\end{tabular*}
\fi

\if\mylang s
\begin{tabular*}{160mm}{lll}
\hline
\multicolumn{3}{l}{\textbf{Författare} \myauthor} \\[1mm] \hline
\multicolumn{3}{l}{\textbf{Titel} \mytitle}  \\[1mm] \hline
\multicolumn{3}{l}{\textbf{Examensprogram} \mydegreeprog} \\[1mm] \hline
\multicolumn{2}{l}{\textbf{Huvudämne} \mymajor} &
\textbf{Huvudämnets kod} \mymajorcode \\[1mm] \hline
\multicolumn{3}{l}{\textbf{Ansvarig lärare} \mysupervisor} \\[1mm] \hline
\multicolumn{3}{l}{\textbf{Handledare} \myinstructor} \\[1mm] \hline
\textbf{Datum} \mypubdate &
\textbf{Sidantal} \mypagenum &
\textbf{Språk} \mylanguage \\[1mm] \hline
\textbf{Sammandrag} \\[1mm] \\[1mm]
\multicolumn{3}{p{155mm}}{%
%%%
%%% START ABSTRACT TEXT (SWEDISH REPORT)
%%%
Här är platsen för sammandrag av ett svenskt specialarbete.
%%%
%%% END ABSTRACT TEXT
%%%
}
\\[1mm] \hline
\multicolumn{3}{l}{\textbf{Nyckelord} \mykeywords} \\[1mm] \hline
\end{tabular*}
\fi

} % end of \sffamily
